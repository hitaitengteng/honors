\documentclass[11pt,twocolumn]{article}

\usepackage{amsmath}
\usepackage{amssymb}

\begin{document}
\title{Analysis of global and regional crop yield growth via genetics-based machine learning}
\author{Will Gantt}
\maketitle

\begin{@twocolumnfalse}
\abstract{As Earth's population continues to grow and as climate change intensifies, it is vital to long-term food security and market stability that we better understand what environmental and economic factors most significantly affect crop yields around the world. To this end, my project applies methods from genetics-based machine learning (GBML) to data collected by Erik Nelson in order to identify correlations between various agricultural inputs and changes in crop yields.}
\end{@twocolumnfalse}

\section{Introduction}
\section{Background} 
\section{Evolutionary Computation and GBML}
Evolutionary computation (EC) comprises a class of algorithms that evolve a population of candidate solutions to a problem over a number of generations. Based loosely on Darwinian evolution, these algorithms exploit processes analogous to those found in biological systems, such as mutation, selection, and genetic crossover, to identify good solutions. EC has proved highly effective in numerous subdomains of artificial intelligence, and is particularly useful when dealing with noisy data and multimodal functions.
\indent Genetics-based machine learning (GBML) applies evolutionary algorithms to major problems in machine learning, and chiefly to problems of optimization and classification. GBML exhibits several distinct advantages over more traditional machine learning approaches. In evolving a population of rules or classifiers, GBML integrates learning and feature selection into a single process. Lastly, it presents a remarkably flexible learning framework that can be applied to a diverse range of problems.
\indent However, GBML algorithms are not without shortcomings, as their time-complexity and run times tend to exceed those of other methods.
\section{Conclusion}
\end{document}